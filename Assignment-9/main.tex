%%%%%%%%%%%%%%%%%%%%%%%%%%%%%%%%%%%%%%%%%%%%%%%%%%%%%%%%%%%%%%%
%
% Welcome to Overleaf --- just edit your LaTeX on the left,
% and we'll compile it for you on the right. If you open the
% 'Share' menu, you can invite other users to edit at the same
% time. See www.overleaf.com/learn for more info. Enjoy!
%
%%%%%%%%%%%%%%%%%%%%%%%%%%%%%%%%%%%%%%%%%%%%%%%%%%%%%%%%%%%%%%%


% Inbuilt themes in beamer
\documentclass{beamer}

% Theme choice:
\usetheme{CambridgeUS}

% Title page details: 
\title{Assignment 9} 
\author{Varun Gupta \\ cs21btech11060}
\date{\today}
\logo{\large \LaTeX{}}


\begin{document}

% Title page frame
\begin{frame}
    \titlepage
\end{frame}

% Remove logo from the next slides
\logo{}


% Outline frame
\begin{frame}{Outline}
    \tableofcontents
\end{frame}
\section{Papoulis Solutions}
\begin{frame}{Problem}
    \begin{block}{Example 7.11}
        The random variables $x_1$ and $x_2$ are jointly normal with zero mean. Determine their conditional density f($x_2 \mid x_1$).
    \end{block}
\end{frame}
\begin{frame}{Solution}
    We know that if the random variables s, $x_1$, ... , $x_{n}$ are jointly normal with zero mean, the linear and nonlinear estimators of s are equal:
    \begin{align}
        \hat{s} = a_1 x_1 + .. + a_{n} x_{n} = g(X) = E\{s \mid X\}\\
        f(s \mid x_1, ..., x_{n}) = \frac{1}{\sqrt{2\pi P}}e^{\frac{-[s-(a_1x_1+...+a_{n}x_{n})]^2}{2P}}
    \end{align}
    therefore,
    \begin{align}
        E\{x_2 \mid x_1\} = a x_1
    \end{align}
    where, $a = \frac{R_{12}}{R_{11}}$\\
    Also,
    \begin{align}
        \sigma^2_{x_2 \mid x_1} = P = E\{(x_2-a x_1)x_2\}=R_{22}-aR_{12}
    \end{align}
\end{frame}
\begin{frame}{Solution}
        Therefore,
        \begin{align}
            f(x_2 \mid x_1) = \frac{1}{\sqrt{2\pi P}}e^{\frac{-(x_2-a x_1)^2}{2P}}
        \end{align}
\end{frame}
\end{document}